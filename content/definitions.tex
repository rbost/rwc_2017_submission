% !TEX root = ../sec-perf-compromise-sse.tex


\section{Definitions and Tools} % (fold)
\label{sec:defs}

Let us start by giving a few definitions that will be useful to formally define searchable encryption and to correctly state theorems.

\subsection{Preliminaries} % (fold)
\label{sub:preliminaries}

In the paper, $\secpar$ is the security parameter and $\negl$ denotes a negligible function in the security parameter.
% Our construction uses pseudo-random functions (PRF) and (keyed) hash functions, for which we use the standard security definitions~\cite{goldreich2004foundations}.
We only consider (possibly probabilistic)  algorithms and protocols running in time polynomial in the security parameter $\lambda$.
In particular, adversaries are probabilistic polynomial time algorithms.

For a finite set $X$, $x \getsr X$ means that $x$ was uniformly sampled from $X$.

% subsection preliminaries (end)

\subsection{Formalism of Symmetric Searchable Encryption} % (fold)
\label{sub:se_formalism}



A \emph{database} $\DB$ is defined as:
\[
\DB = \left\{ (\ind_i,\W_i) : 1 \leq i \leq D \right\},
\]
with \( \ind_i \in \zo^\ell, \W_i \subseteq \zo^* \) 
and where $\ind_i$ are distinct \emph{document indices}, represented by $\ell$-bit strings, and $\W_i$ is a set of \emph{keywords} matching document $\ind_i$, represented by binary strings of arbitrary length.
Note that, in this manuscript, documents are identified with their indices.
In addition, let us define:
% \begin{align*}
% 	D &  = |\DB|  \text{ the number of documents;}\\
% 	\W & = \bigcup_{i=1}^D \W_i \text{ the set of keywords;}\\
% 	K &  = |\W| \text{ the number of keywords;}\\
% 	N &  = \sum_{i=1}^D |\W_i|  \text{ the number of document/keyword pairs.}
% \end{align*}
\\
\centerline{
\renewcommand{\tabcolsep}{.7mm}
\begin{tabular}{rl}
$D$ & $ = |\DB|$  the number of documents;\\
$\W$ & $ = \cup_{i=1}^D \W_i$ the set of keywords;\\
$K$ & $ = |\W|$ the number of keywords;\\
$N$ & $ = \sum_{i=1}^D |\W_i|$ the number of document/keyword pairs.
\end{tabular}
}
Let $\DB(w)$ denote the set of documents containing keyword $w$:
\[
\DB(w) = \left\{\ind_i | w \in \W_i \text{ and } (\ind_i, \W_i) \in \DB \right\}.
\]
The value $n_w$ is the number of documents matching $w$: $n_w = |\DB(w)|$.

A \emph{dynamic symmetric searchable encryption (SSE) scheme} is a triple $\Sigma = (\Setup, \allowbreak \Search,  \allowbreak  \Update)$ consisting of one algorithm and two protocols between a client and a server:
\begin{itemize}
	\item $\Setup(\DB)$ is a probabilistic algorithm that takes as input the initial database $\DB$. 
	It outputs a triple $(\EDB, K_\Sigma, \sigma)$, where $K_\Sigma$ is the master secret key, $\EDB$ is an encrypted database, and $\sigma$ is the client's state.
	
	\item $\Search(K_\Sigma,q,\sigma;\EDB) = (\Search_C(K_\Sigma,q,\sigma),\Search_S(\EDB))$ is a protocol between the client with input the master secret key $K_\Sigma$, the client's internal state $\sigma$, and a search query $q$; and the server with input the encrypted database $\EDB$.
	
	After completing the $\Search$ protocol, the client outputs a list $R$ of results and a new internal state $\sigma'$.
	Both $R$ and $\sigma'$ can take the special value $\bot$ to signify an error or a failure in the execution of the protocol.
	The server possibly outputs an updated encrypted database $\EDB'$.
	
	
	The query $q$ can be of any kind: although we will mainly focus on search queries restricted to a single-keyword $w$ (and hence often identify $q$ with $w$), this formalism also supports range queries or boolean queries --- a search query consisting of a boolean formula $\phi$ and a set of keywords $(w_1,\dots,w_n)$ matching the set of documents $\DB(\phi, w_1,\dots,w_n)$ such that
	\begin{align*}
		\DB(\phi, w_1,\dots,w_n) = \{\ind_i | \phi(b_1, \dots, b_n) = \true, \text{ where } b_j &= (w_j \in \W_i), \\
		 \text{ and } &(\ind_i, \W_i) \in \DB \}.
	\end{align*}
	More generally, $\DB(q)$ denotes the set of documents matching the query $q$.
				
	\item $\Update(K_\Sigma,\sigma, \op,\opin;\EDB)=(\Update_C(K_\Sigma,\sigma, \op,\opin), \allowbreak \Update_S(\EDB))$ is a protocol between the client with input the key $K_\Sigma$ and internal state $\sigma$, an operation $\op$, and an input $\opin$ for the operation; and the server with input $\EDB$.
	Again, this formalism covers a wide range of different update operations, such as merges, duplications, \emph{etc}.
	Yet, this thesis focuses on simpler operations: the update operations are taken from the set $\{\add, \allowbreak\del\}$, meaning, respectively, the addition and the deletion of a set of keywords to a document.
	The input $\opin$ is thus parsed as an index $\ind$, pointing to the modified document, and a set $W$ of keywords to insert or delete.


	At the end of the execution of the protocol, the client outputs a new state $\sigma'$, which can take the special value $\bot$, and the server outputs a new encrypted database $\EDB'$.
\end{itemize}

In this formalism, we explicitly separate the client's state and the key.
Informally, we want the key to be fixed while the state is mutable.
We do not require a scheme to have a state (beyond the key), and when it is empty, we may omit is from the protocol's signature.
Also, we do not require the size of the state to be upper bounded, \eg by the number of keywords. 
For example, we could imagine a scheme working only on the client side, with no storage on the server.
This would be a perfectly valid (and secure), but very costly scheme. 
Hence, the size of the client's state is an important parameter regarding the compromise between security and performance.

An important restriction of this definition is \emph{static symmetric searchable encryption}.
Such schemes do not support update requests, and hence do not implement the $\Update$ protocol.
In that case, once the encrypted database has been set up, it is immutable.
% subsection se_formalism (end)

\subsection{Confidentiality of Searchable Encryption} % (fold)
\label{sub:se_security}

	Two definitions exist for the confidentiality of searchable encryption. One is indistinguishability-based, and the other simulation-based.
	In this paper, we will only use the indistinguishability-based defintion, which is weaker than the simulation-based one (\cf \cite{CCS:CGKO06}).
	
	The definition is parametrized by a \emph{leakage function} $\L = (\L^\Stp,\L^\Srch,\L^\Updt)$ describing what the protocol leaks to the adversary, and formalized as a stateful algorithm.
	The definition ensures that the scheme does not reveal any information beyond the ones that can be inferred from the leakage function.

The $\SSEInd$ game picks a random bit $b$ and the adversary's goal is to guess $b$.
To do so, he (adaptively) submits two sequences of a database and $n$ queries $(\DB^0,q^0_1,\dots,q^0_n)$ and $(\DB^1,q^1_1,\dots,q^1_n)$, and receives from the game the encrypted database and the transcript of the queries corresponding the sequence  $(\DB^b,q^b_1,\dots,q^b_n)$.
The sequences submitted by the adversary must satisfy a very important constraint: the leakage must be the same for both of them.
Otherwise, the game aborts as soon as the adversary submits two queries (or two databases) having a different leakage.
Finally, the adversary outputs a bit $b'$, and wins the game if he successfully guesses $b$.

\subimport{figs/}{sse_ind_honest.tex}

\begin{definition}[Indistinguishability-based confidentiality of SSE]
	\label{def:sse_ind_conf}
	Let $\Sigma$ be an SSE scheme.
	For an adversary $A$, the advantage $\AdvSSEInd{\Sigma,\L,A}$ of $A$ in the indistinguishability-based confidentiality game is defined as
	\[
	\AdvSSEInd{\Sigma,\L,A} = \left|\frac{1}{2} - \Prob[\SSEInd^A_{\Sigma,\L}(\secpar) = 1 ] \right|.
	\]
	An SSE scheme $\Sigma$ is \emph{$\L$-adaptively-indistinguishability secure} if for all polynomial-time adversary $A$, $\AdvSSEInd{\Sigma,\L,A}$  is negligible in $\secpar$.
\end{definition}


% subsection se_security (end)

\subsection{Soundness of Searchable Encryption} % (fold)
\label{sub:se_soundness}

The last important security definition we will use is about the notion of \emph{soundness}.
As for regular soundness definitions, \emph{e.g.} for provers, it describes the fact that no adversary can cheat the client, and make him accept incorrect search results. 

To give a proper soundness definition, we use the game $\SSESnd$ defined in Figure~\ref{fig:sound_sse}.
The game closely follows the game used to define soundness of interactive provers~\cite{BOOK:Goldreich04}: the client must not accept an invalid search result.
Also, the dynamism of the database raises a difficult point: the verification has to be done over the current version of the database, and this one must not be modifiable undetectably by a malicious server. 
Hence, $\SSESnd$ does not apply the update operation on the database when the client rejects the execution of the $\Update$ protocol with the server. 

\subimport{figs/}{sse_soundness.tex}

\begin{definition}[SSE Soundness]
	Let $\Sigma$ be an SSE scheme.
	For an adversary $A$, the advantage $\AdvSSESound{\Sigma,A}$ of $A$ in the soundness game is defined as
	\[
	\AdvSSESound{\Sigma,A} = \Prob[\SSESnd^A_\Sigma(\secpar) = 1 ].
	\]
	An SSE scheme $\Sigma$ is \emph{sound} if for all polynomial-time adversary $A$, $\AdvSSESound{\Sigma,A}$  is negligible in $\secpar$.
\end{definition}

This definition can be seen as a generalization of the reliability definition of Kurosawa and Ohtaki~\cite{FC:KurOht12}: in that paper, the authors study the case of single-roundtrip constructions (also static ones, but they later extended their definitions to the dynamic setting in~\cite{CANS:KurOht13}).

% subsection se_soundness (end)

% section defs (end)