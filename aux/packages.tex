% !TEX root = ../sec-perf-compromise-sse.tex

% Use utf-8 encoding for foreign characters
\usepackage[utf8]{inputenc}
\usepackage[T1]{fontenc}

\usepackage{pdfpages} % For front and back covers, warning, when scehack is used, must be loaded before mathtools

\usepackage{amsmath,amsfonts,amssymb}

\usepackage{mathtools}
\usepackage{mathdots}
\usepackage{stmaryrd}

\usepackage{enumitem, framed}

%% For gorgeous subfigures
\usepackage{caption}
\captionsetup{justification=justified,labelfont=bf,labelsep=endash}
% \usepackage{subcaption}
\usepackage{dblfloatfix}                 


\usepackage{multicol}
\usepackage{boxedminipage}


%----------------------------------------------------------------------
%                     Floats and Tables
%----------------------------------------------------------------------

\usepackage{booktabs}

%% For footnotes in tables
\usepackage{footnote}
\makesavenoteenv{tabular}
\makesavenoteenv{table}
\makesavenoteenv{table*}
\usepackage{multirow}

\usepackage{rotating}



%the following two lines are to correct the default hyphenation for timestamp (which is somehow times-tamp)
\usepackage{hyphenat}
\hyphenation{time-stamp}


%----------------------------------------------------------------------
%                     Todonotes
%----------------------------------------------------------------------

\iftoggle{allowtodo}{
  \iftoggle{showtodo}{
  	\iftoggle{inlinetodo}{
		\newcommand{\todo}[1]{\textcolor{red}{[{\footnotesize {\bf TODO:} { {#1}}}]}}
		
	}{
		\usepackage[textsize=small]{todonotes}
	}
  }{
    \usepackage[disable]{todonotes}
	\renewcommand{\todo}[1]{}
  }
  % add xspace to todo command (http://tex.stackexchange.com/a/68741)
  \usepackage{xspace}
  % \makeatletter
  \expandafter\apptocmd\csname\string\todo\endcsname{\xspace}{}{}
  % \makeatother
}{
}

%----------------------------------------------------------------------
%                     Algorithms
%----------------------------------------------------------------------

\usepackage{algorithm}
\usepackage{algorithmicx}
\usepackage[]{algpseudocode}
\newcommand{\algsep}{\begin{center}\rule{\textwidth}{0.4pt}\end{center}}


\input{aux/algorithms.tex}


\usepackage{float}

\newfloat{protocol}{tbph}{lop}
\floatname{protocol}{Protocol}


%----------------------------------------------------------------------
%                     Theorems
%----------------------------------------------------------------------

\usepackage{amsthm}


\newtheorem{theorem}{Theorem}

\newtheorem{proposition}[theorem]{Proposition}
% \newtheorem{informaltheorem}[theorem]{Theorem}
% \newtheorem{informalcorollary}[theorem]{Corollary}
% \newtheorem{informalclaim}[theorem]{Claim}
\newtheorem{claim}[theorem]{Claim}
% \newtheorem{remark}{Remark}[section]
% \newtheorem{remark}[theorem]{Remark}
\newtheorem{lemma}[theorem]{Lemma}
\newtheorem{corollary}[theorem]{Corollary}
% \newtheorem{fact}[theorem]{Fact}
% \newtheorem{assumption}{Assumption}
\newtheorem{definition}{Definition}[section]


% To repeat theorems numbers easily (e.g. when deferring the proof)
\makeatletter
\newtheorem*{rep@theorem}{\rep@title}
\newcommand{\newreptheorem}[2]{%
\newenvironment{rep#1}[1]{%
 \def\rep@title{#2 \ref{##1}}%
 \begin{rep@theorem}}%
 {\end{rep@theorem}}}
\makeatother

\newreptheorem{theorem}{Theorem}
\newreptheorem{proposition}{Proposition}

%----------------------------------------------------------------------
%                     Crypto
%----------------------------------------------------------------------

\usepackage[lambda,advantage,adversary,landau,sets,notions,logic,ff,primitives,events,asymptotics,keys]{cryptocode}

% Advandtage style
\renewcommand{\pcadvantagesuperstyle }[1]{\mathrm{#1}}
% \renewcommand{\pcadvantagesubstyle }[1]{#1}

% Adversary style
% \renewcommand{\pcadvstyle }[1]{\ mathcal{#1}}

%----------------------------------------------------------------------
%                     TikZ
%----------------------------------------------------------------------


\usepackage{tikz}
\usetikzlibrary{shapes,positioning,calc}
\usetikzlibrary{arrows}
\usetikzlibrary{fit}

%----------------------------------------------------------------------
%                     GnuPlot (with TikZ)
%----------------------------------------------------------------------
% \usepackage{gnuplot_sty/gnuplot-lua-tikz}


%----------------------------------------------------------------------
%                     Import (useful when using a lot of files)
%----------------------------------------------------------------------
\usepackage{import}

%----------------------------------------------------------------------
%                     Figures and Units
%----------------------------------------------------------------------
\usepackage[binary-units,mode=text]{siunitx}

%----------------------------------------------------------------------
%                     Hyperref at last
%----------------------------------------------------------------------

\definecolor{blue_link}{RGB}{0,0,150}

\usepackage[pageanchor=false,pdfpagelabels=true]{hyperref}

\hypersetup{
  % linktoc        = page,
  % pdfpagemode    = UseOutlines,
  colorlinks     = \iftoggle{printversion}{false}{true},
  linkcolor      = blue_link,
  citecolor      = blue_link,
  urlcolor       = blue_link,
  bookmarksdepth = 3
}

\makeatletter
\AtBeginDocument{
  \hypersetup{
    pdftitle  = {\@title},
    pdfauthor = {\@author}
  }
}
\makeatother